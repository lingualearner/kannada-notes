\documentclass{article}
\usepackage[a4paper,portrait,margin=1in]{geometry}
\usepackage{graphicx}
\usepackage{fontspec}
\newfontfamily\s[Script=Devanagari]{Shobhika}
\newfontfamily\sa[Script=Devanagari]{Shobhika-Bold}
\newfontfamily\hindifont{Noto Sans Devanagari}[Script=Devanagari]
\usepackage{polyglossia}
\usepackage{fontawesome}
\setmainlanguage{english}
\setotherlanguages{marathi,hindi,sanskrit}
\newfontfamily\devanagarifont[Script=Devanagari]{Noto Serif Devanagari}
\usepackage{skt}
\usepackage{multicol}



\usepackage[most]{tcolorbox}

\usepackage[utf8]{inputenc}
\usepackage[T1]{fontenc}
\usepackage{verse}
\settowidth{\versewidth}{It lies behind stars and under hills,}
\addtolength{\versewidth}{2em}
\usepackage{xcolor}





\title {Sanskrit typing in Overleaf}
\author{Priyanka Salekar}
\date{January 2024}

\begin{document}

\newcommand{\frase}[1]{
    \textbf{\textcolor{teal}{#1}}
}
\newcommand{\frasedos}[1]{
    \textbf{\textcolor{cyan}{#1}}
}
\newcommand{\frasetres}[1]{
    \textbf{\textcolor{teal}{#1}}
}

\newcommand{\tipo}[1]{
    \fcolorbox{yellow}{yellow!50}{#1}
}
\newcommand{\subtipo}[1]{
    \fcolorbox{yellow}{yellow!20}{#1}
}
\newcommand{\tipodos}[1]{
    \fcolorbox{cyan!50}{cyan!10}{#1}
}
\newcommand{\tipotres}[1]{
    \fcolorbox{teal}{teal!20}{#1}
}

\newcommand{\ejemplobox}[1]{
\begin{tcolorbox}
[breakable,colback=white,colframe=cyan,width=\dimexpr\textwidth+12mm\relax,enlarge left by=-6mm]
#1
\end{tcolorbox}
}


\maketitle
\vspace{20mm}
{\begin{center}
 \large\textbf{\textit{All fonts work with XeLaTex compiler only.\\ Otherwise upload the script file separetely.}}   
\end{center}}
\pagebreak
\section{क्रिया}

\subsubsection{ \s \tipo{बा/बन्नि} $\longrightarrow$ आ/आओ/come/venir}
\begin{enumerate}
    \item {\s इल्लि \frase{बा} } $\longrightarrow$ {\s इधर \frasedos{आ}}
    \item {\s नन्ना जोतगागे \frase{बन्निि} $\longrightarrow$  मेरे साथमे \frasedos{आओ }  }
    \item {\s उटक्के \frase{बन्नि} $\longrightarrow$  खानेको \frasedos{आओ }  }    
\end{enumerate}

\subsubsection{ \s \tipo{होगु/होगि} $\longrightarrow$ जा/जाओ/go/ir }
\begin{enumerate}
    \item {\s शालेगे \frase{होगु}  } $\longrightarrow$ {\s  स्कूल को \frasedos{जाओ}}
    \item {\s अल्लिगे \frase{होगबेड़ा} $\longrightarrow$  यहाँ से \frasedos{मत​ जाओ} }
    \item {\s मनेगे \frase{होगु} $\longrightarrow$  घरको \frasedos{जाओ}  }    
\end{enumerate}

\subsubsection{ \s \tipo{होगु/होगि} $\longrightarrow$  जा/जाओ/go/ir }
\begin{enumerate}
    \item {\s शालेगे \frase{होगु}  } $\longrightarrow$ {\s  स्कूल \frasedos{जाओ}}
    \item {\s अल्लिगे \frase{होगबेड़ा} $\longrightarrow$  यहाँ से \frasedos{जाओ} }
    \item {\s मनेगे \frase{होगु} $\longrightarrow$  घरको \frasedos{जाओ}  }    
\end{enumerate}

\subsubsection{ \s \tipo{इर्तिनि} $\longrightarrow$  रहो/go/ir }
\begin{enumerate}
    \item {\s नीवु नन जोत्थे इल्लि \frasedos{इरु}  $\longrightarrow$   मेरे साथ यहा  \frasedos{रहो} }
    \item {\s नानु इल्लि \frase{ईरुत्तेने} $\longrightarrow$ मै यहा \frasedos{रेहता} हु}
\end{enumerate}




\section{\s क्रिया + बेकु (Please do/ Imperative Mood)}
\begin{multicols}{2}
\begin{enumerate}
    \item {\s \frase{माड} + \frasedos{बेकु} $\longrightarrow$ \s  करो } 
    \item {\s \frase{कोड} + \frasedos{बेकु} $\longrightarrow$  दो  } 
    \item {\s \frase{इडा} + \frasedos{बेकु} $\longrightarrow$  रखो  } 
    \item {\s \frase{तगो} + \frasedos{बेकु} $\longrightarrow$ खरीदो }
    \item {\s \frase{माताळ} + \frasedos{बेकु} $\longrightarrow$ बात  करो }
    \item {\s \frase{हेळ} + \frasedos{बेकु} $\longrightarrow$  बोलो }
    \item {\s \frase{केऴ} + \frasedos{बेकु} $\longrightarrow$  सुनो }
    \item {\s \frase{निल्लिस} + \frasedos{बेकु} $\longrightarrow$ रुको }
    \item {\s \frase{बर} + \frasedos{बेकु} $\longrightarrow$ आओ }
    \item {\s \frase{होग} + \frasedos{बेकु} $\longrightarrow$ जाओ }        
\end{enumerate}
\end{multicols}

\section{\s क्रिया + बेडि (don't)} 
\begin{multicols}{2}

\begin{enumerate}
    \item {\s \frase{माड} + \frasedos{बेडि} $\longrightarrow$ \s मत करो } 
    \item {\s \frase{कोड} + \frasedos{बेडि} $\longrightarrow$ मत दो  } 
    \item {\s \frase{इडा} + \frasedos{बेडि} $\longrightarrow$ मत रखो  } 
    \item {\s \frase{तगो} + \frasedos{बेडि} $\longrightarrow$ मत लो /मत खरीदो }
    \item {\s \frase{माताळ} + \frasedos{बेडि} $\longrightarrow$ बात  मत करो }
    \item {\s \frase{हेळ} + \frasedos{बेडि} $\longrightarrow$ मत बोलो }
    \item {\s \frase{केऴ} + \frasedos{बेडि} $\longrightarrow$ मत सुनो }
    \item {\s \frase{निल्लिस} + \frasedos{बेडि} $\longrightarrow$ मत रुको }
    \item {\s \frase{बर} + \frasedos{बेडि} $\longrightarrow$ मत आओ }
    \item {\s \frase{होग} + \frasedos{बेडि} $\longrightarrow$ मत जाओ }
    \item {\s \frase{होग} + \frasedos{बेडि} $\longrightarrow$ मत जाओ }
    
\end{enumerate}
\end{multicols}

\section{\s क्रिया + बारदु (Should not)}


\begin{enumerate}
    \item {\s \frase{माड} + \frasedos{बारदु} $\longrightarrow$ \s नही करना चाहिये  (should not do)}
    \item {\s \frase{हाक} + \frasedos{बारदु} $\longrightarrow$ नहि रखना चाहिये (should not put)  } 
    \item {\s \frase{इडा} + \frasedos{बारदु} $\longrightarrow$ नहि रखना चाहिये (should not keep) } 
\end{enumerate}

\section{others}
\begin{enumerate}
    \item {\s  माड + कोडि $\longrightarrow$ करके दो (do) }
    \item {\s  तगोन्ड कोडि $\longrightarrow$ लाकर दो (bring) }
    \item 

\end{enumerate}


\section{\s  sentence join examples }

\begin{enumerate}
    \item  {\s ननगे पैन​ कोडी याकान्द्रे नानु केलसा माडबेकु $\longrightarrow$ मुझे पैन​ दो \frasedos{क्योकि} मुझे काम करना है } 
    \item {\s येल्ला Documents इद्रु \frase{आन्द्रे} याके नन केलसा माडता इल्ला $\longrightarrow$ हालांकि सारे Documents  \frasedos{पर​} मेरा काम नहि कर रहे}
\end{enumerate}


\section{\s क्रिया + इद्रे (if you, करोगे तो  )}
\begin{enumerate}
    \item {\s \frase{माडिदरे} $\longrightarrow$ \s करोगे तो  } 
    \item {\s \frase{तगोन्ड्रे} $\longrightarrow$ \s लोगे तो / खरीदोगे तो  } 
    \item {\s \frase{कोट्टरे}  $\longrightarrow$ \s दोगे तो } 
    \item {\s \frase{होगरे}  $\longrightarrow$ \s जाओगे तो } 
    \item {\s \frase{बंदरे}  $\longrightarrow$ \s  आओगे } 
    
\end{enumerate}

\section{\s क्रिया + अक्के / infinitive form or for (के लिये / ने का)}
\begin{enumerate}
    \item {\s \frase{माडक्के} $\longrightarrow$ \s करने के लिये/ करने का  } 
    \item {\s \frase{केलसक्के} $\longrightarrow$ \s काम करने के लिये/ काम का  } 

\end{enumerate}

\section{\s क्रिया + इदिया / (non people or इदु / अदु)}
\begin{enumerate}
    \item {\s \frase{माडिदिया} $\longrightarrow$ \s कर रहा है क्या? } 
    \item {\s \frase{कोडतिया} $\longrightarrow$ \s दे रहा है क्या ? } 
\end{enumerate}


\section{\s संज्ञा + आगि (ly)}
\begin{multicols}{2}

\begin{enumerate}
    \item {\s \frase{सरि} + \frasedos{आगि} $\longrightarrow$ \s सही से } 
    \item {\s \frase{चन्ना} + \frasedos{आगि} $\longrightarrow$ \s अच्छे से } 
    \item {\s \frase{clean} + \frasedos{आगि} $\longrightarrow$ \s cleanly} 

    \item {\s \frase{निधानवागि} $\longrightarrow$ \s धीरे से } 
    \item {\s \frase{वेगवागि} $\longrightarrow$ \s जल्दी से } 
\end{enumerate}
\end{multicols}

\section{\s संज्ञा + इंदा (ly)}
\begin{multicols}{2}

\begin{enumerate}
    \item {\s \frase{नाळेयिन्दा} $\longrightarrow$ \s कल से } 
    \item {\s \frase{नाळेयिन्दा} $\longrightarrow$ \s कल से } 
\end{enumerate}
\end{multicols}


\section{\s बेरे + संज्ञा  (other )}


\begin{enumerate}
    \item {\s \frase{बेरे } + \frasedos{car} $\longrightarrow$ \s other car } 
    \item {\s \frase{बेरे} + \frasedos{याव} $\longrightarrow$ \s (which other) और कोंनसी } (yaava always goes with bere) 
    \item {\s \frase{चन्ना} + \frasedos{आगि} $\longrightarrow$ \s अच्छे से } 
\end{enumerate}

\section{\s Time phrases}


\begin{enumerate}
    \item {\s \frase{वारक्के ओन्दु सरि} $\longrightarrow$ \s हफ्ते मे एक बार } 
    \item {\s \frase{तिंगळिगे ओन्दु सरि} $\longrightarrow$ \s महिने मे एक बार } 
    \item {\s \frase{ओन्दु सरि} $\longrightarrow$ \s एक बार } 
    \item {\s \frase{तप्पिसदे} $\longrightarrow$ \s भुले बिना } 

\end{enumerate}


\begin{enumerate}
    \item {\s \frase{आन्टे}  $\longrightarrow$ \s वह आया इसलिए हम गए। } 
    \item {\s \frase{आन्द्रे}  $\longrightarrow$ मत दो  } 
    \item {\s \frase{इडा} + \frasedos{बेडि} $\longrightarrow$ मत रखो  } 
    \item {\s \frase{तगो} + \frasedos{बेडि} $\longrightarrow$ मत लो /मत खरीदो }
    \item {\s \frase{माताळ} + \frasedos{बेडि} $\longrightarrow$ बात  मत करो }
    \item {\s \frase{हेळ} + \frasedos{बेडि} $\longrightarrow$ मत बोलो }
    \item {\s \frase{केऴ} + \frasedos{बेडि} $\longrightarrow$ मत सुनो }
    \item {\s \frase{निल्लिस} + \frasedos{बेडि} $\longrightarrow$ मत रुको }
    \item {\s \frase{बर} + \frasedos{बेडि} $\longrightarrow$ मत आओ }
    \item {\s \frase{होग} + \frasedos{बेडि} $\longrightarrow$ मत जाओ }
    \item {\s \frase{होग} + \frasedos{बेडि} $\longrightarrow$ मत जाओ }
    
\end{enumerate}

\section{\s Important Questions  } 
\begin{enumerate}
    \item {\s इल्लि \frase{बेरे याव} मनेयल्लि केलसा माडतिया $\longrightarrow$ यहाँ \frasedos{और कौनसे}  घर मे काम करते हो }
    \item {\s office work timings येनु $\longrightarrow$ what are office working time?}
    \item {\s  येन येनु फ़िल माडबेकु $\longrightarrow$ \frasedos{what all} do I have to fill}
    \item {\s निवु टोकन \frase{एष्टू हॉट्टिगे} कोडक्के स्टार्ट माड्रातीरा$\longrightarrow$ आप टोकन \frasedos{कब तक}​ देना करना शुरू करेंगे?}
\end{enumerate}

\section{\s अगिदु । अयितु } 
\begin{enumerate}
    \item {\s उटा \frase{आगिदे}  $\longrightarrow$  खाना \frasedos{हो गया है}}
    \item {\s केलसा \frase{आगिदे}  $\longrightarrow$  काम \frasedos{हो गया है}}
    \item {\s Rain \frase{आगता}  इदे $\longrightarrow$ Rain \frasedos{हो रही है}} 
    \item 

\end{enumerate}


\section{\s Important words } 

\begin{enumerate}
    \item {\s अवनु बन्दनु \frase{अन्टे} नावु होदेवु $\longrightarrow$ वह आया \frasedos{इसलिए} हम गए }
    \item {\s \frase{होसा} पुस्तक तगोन्डकोडि $\longrightarrow$ \frasedos{नई} पुस्तक लाकर दो  }
    \item {\s ननगे \frase{अष्टे} बेकु $\longrightarrow$ मुझे \frasedos{इतना हि} चाहिये  }
    \item {\s नानु ई मुवी नोडि \frase{याक अन्दरे} अदु तुम्बा चनागिदे $\longrightarrow$ मैने ये मुवी देखी \frase{क्युकि} वो बहुत अच्छी है}
    \item {\s बस्सु \frase{आगता} इदे / बस्सु \frase{बरता} इदे   } $\longrightarrow$ {\s बस \frasedos{आ गई} है }
    \item {\s \frase{इन्नोन्द सला} माडबेकु  $\longrightarrow$ \frasedos{ दूसरी बार} करना होगा}
    \item {\s \frase{तप्पिसदे} बरबेकु $\longrightarrow$  \frasedos{बिना भुले} आना होगा}
    \item {\s \frase{तप्पिसदे} फ़ोर्म अप्लाय माडबेकु $\longrightarrow$  \frasedos{बिना भुले} फ़ोर्म अप्लाय करना होगा }
    \item {\s Bangaloreyalli येल्ला कडे Traffic  इदे $\longrightarrow$ Bangalore में  \frasedos{हर जगाह } Traffic है }
    \item {\s येल्ला Documents \frase{इद्रु} आन्द्रे याके नन केलसा माडता इल्ला $\longrightarrow$ \frasedos{हालांकि} सारे Documents  \frasedos{पर​} मेरा काम नहि कर रहे}
    \item {\s निवु टोकन \frase{एष्टू हॉट्टिगे} कोडक्के स्टार्ट माड्रातीरा$\longrightarrow$ आप टोकन \frasedos{कब तक}​ देना करना शुरू करेंगे?}
    \item {\s \frase{ईगले} बन्नि $\longrightarrow$ \frasedos{अभी ही} आओ $\longrightarrow$ Come \frasedos{right now}}
    \item {\s \frase{ईगा} बन्नि $\longrightarrow$ \frasedos{अभी } आओ $\longrightarrow$ Come \frasedos{ now}}
\end{enumerate}

\section{\s  verb examples (Normal)} 

\begin{enumerate}
    \item {\s ओन्दु एप्लिकेशन फ़ोर्म \frase{ कोडतिरा }?$\longrightarrow$ एक अप्प्लिकेशन फ़ोर्म \frasedos{दोगे} ? }
    \item {\s ताजा तरकारी  \frase{सिगत्ते} ?$\longrightarrow$ ताजा सब्जि \frasedos{ लाना }}
    \item {\s टोकन कोडक्के स्टार्ट माडतिरा }
\end{enumerate}


\section{Conversations}

\frase{\s A :  नीवु येल्लि होगताइद्दीरा ? }\\
\\
\frasedos{\s B नानु मौल अल्लि होगताइद्दीनि } \\
\\
\frase{\s A : ई माल तुम्बा दोड्डा इदे} \\
\\
\frasedos{\s B : निवु मुविगे टिकिट बुक माडिद्दिरा }\\
\\
\frase{\s A :  नन हत्थिरा मुरु टिकिट इवे }\\
\\
\frasedos{\s B :  मुवि मोदलु शोपिग माडना  }\\
\\
\frase{\s A : ननगे इ ड्रेस ईष्टा}\\
\\
\frasedos{\s B : इ ड्रेस चनागिल्ला । बेरे ड्रेस तगो } \\
\\
\frase{\s A : इ सारी यावा  ब्रान्ड आगिदे ? }\\
\\
\frasedos{\s B : इ ड्रेस चनगिदे , नीवु तगोतिरा ? }\\
\\
\frase{\s A : हवधु , ननगे कोडु}\\
\\
\frasedos{\s B :  ईगा नावु होगना }\\


\section{Varamahalakshmi}
\frase{\s A : इवत्तु वरमाहालक्ष्मीगे देवस्थानअक्के होगबेकु } $\longrightarrow$ {\s आज वरमहालक्ष्मी के लिए मंदिर जाना है } \\
\\
\frasedos{\s B : निवु अंगडिगे होगि मत्तु flowers तगोन्डबन्नि / तन्नी } \\
\\
\frase{\s A : सविता, इ रुम चनागिल्ला |  इन्नोद सला clean माडबेकु  } \\
\\
\frasedos{\s B : पुजा सामग्रिगळु येल्लि इवे ?}\\
\\
\frase{\s A : ननगे इदु गोत्थिल्ला ?}\\
\\
\frasedos{\s A : ओ देवरा । ननगे सहाय माडी ? नन्ना मातु केळि . इल्लि यारादरु इद्दारे ? }\\
\\
\frase{\s B : चिन्ते मडबेडा | येल्लारु इल्लिगे बन्नि | नावु start माडना }

\section{Introductions}
\frase{\s A : नमस्कारा, येन समाचारा, निम्मा मने अल्लि येल्लारु जना चेन्नागि इदया?}
\\
\frasedos{\s B : हवधु, नन्न मने सर्वे जना चनागिदे . निवु येन​ केलसा माडिदिया? निवु IT engineer? } \\
\\
\frase{\s A : हवधु । नानु एजिंनियर । नानु coding माडक्के इष्टा | निमन्नु मधवे आगिदिया ?   } \\
\\
\frasedos{\s B : हवधु , नमन्नु  इब्बरु हुडुगारु अन्द्रे ओन्दु हुडुगि }\\
\\
\frase{\s A : हवुधा? आदक्के तुम्बा कोल बन्दिद्रु }\\
\\
\frasedos{\s A : ई कोडू , आ कोडू ,नन  जिवनअल्लि तुंबा तलेनुवु इदे । येन माडबेकु  }\\
\\
\frase{\s B :चिन्ते माडबेडा , निम मगु येल्लि कलक्के होग्तिनि }\\
\\
\frasedos{\s B : नन चिक्क मगा स्कूलअल्लि होगताइदारे, नन दोड्ड मगा कोलेजल्लि  course माडताईदारे | अवरु entrence ए‍क्झामअल्लि first रेन्क अगिदे}

\section{\s Important verb examples (Conjugated - होगबेकु) } 

\begin{enumerate}
    \item {\s ओन्दु Application form \frase{ कोडतिरा }?$\longrightarrow$ एक Application form  \frasedos{दोगे} ? }
    \item {\s Glass \frase{ तगोडबन्नि }?$\longrightarrow$ Glass \frasedos{लेकर आओ} ? }
    \item {\s   }
\end{enumerate}

\section{\s  verb examples (Conjugated - अक्के form) } 

\begin{enumerate}
    \item {\s अडिगे\frase{ माडक्के }मुन्चे   cooker wash माडि $\longrightarrow$ \frasedos{रसोई} करने के पेहले cooker wash करना } 
\end{enumerate}



\section{\s  Rules} 

\begin{enumerate}
    \item {\s \frase{ अक्के  } and  $\longrightarrow$ \frasedos{रसोई} करने के पेहले cooker wash करना } 
    \item 
\end{enumerate}



\section{Devanagari - Shobhika font with Phonetic keyboard}
\begin{Large}
    \begin{itemize}
    \item This need keyboard setting changed to Kagapa Phonetic keyboard.
    
    \item Must use these packages - fontspec, script packages
    %\usepackage{fontspec}
    %\newfontfamily\s[Script=Devanagari]{Shobhika}, %\newfontfamily\sa[Script=Devanagari]{Shobhika-Bold}
    \item For e.g. -\\
{\scriptsize {\s हो हौ ऽ हं स श ष कार्स्न्य विद्या सङ्कर्षण}}\\
{\small {\s क्ष ज्ञ ण्य डढ टठ ॠ ॠ ळ ऋ ल ऌ वे ऐ आ अ ई इ उ ऊहु }}\\
{\large {\s उपाध्यायान्दशाचार्याः विमूर्च्छिताः निष्कृतिः}}\\
{\Large {\s क्त्वा क्ततवतु निष्ठा सक्तुमिव तितउना }}\\
{\huge {\s होहौऽहंसशष कार्स्न्य विद्या सङ्कर्षण}} 
\end{itemize}
\end{Large}
\pagebreak

\section{Devanagari - Shobhika font with Google Input}
\begin{Large}
    \begin{itemize}
    \item Needs google input tools installed or you can copy paste the content from its web-based version.\\
    Link - https://www.google.co.in/inputtools/try/
    \item Must use packages - fontspec, script packages
    \item Set language to Sanskrit, Hindi or Marathi for typing Devanagari.
    %\usepackage{fontspec}
    %\newfontfamily\s[Script=Devanagari]{Shobhika}, %\newfontfamily\sa[Script=Devanagari]{Shobhika-Bold}
    \item For e.g. -\\
{\scriptsize {\s अ आ इ ई उ ऊ ऋ ॠ लृ लृ३ ए ऐ ओ औ अं अः ळ क्ष ज्ञ त्र ण्य}}\\
{\small {\s अभून् नृपो विबुधसखः परन्तपः}}\\
{\large {\s श्रुतान्वितो दशरथ इत्युदाहृतः।}}\\
{\Large {\s गुणैर्वरं भुवनहितच्छलेन यं}}\\
{\huge {\s सनातनः पितरमुपागमत् स्वयम्॥}} 
\end{itemize}
\end{Large}
\pagebreak

\section{Devanagari - Regular font with Google input}
\begin{Large}
    \begin{itemize}
    \item Change keyboard settings to Google input tool or can copy paste from Aksharmukh - https://aksharamukha.appspot.com/converter

    \item Must use hindifont command before every writing. 
    %{\hindifont[]{}{}}%
    
    \item Must use these packages - fontspec, newfont command
    %\newfontfamily
    %\hindifont{Noto Sans Devanagari}[Script=Devanagari]
     \item For e.g. -\\
     {\hindifont
     {\scriptsize क्ष ज्ञ ण्य डढ टठ ॠ ॠ ळ ऋ ल ऌ वे ऐ आ अ ई इ उ ऊहु}\\
     {\small उपाध्यायान्दशाचार्याः विमूर्च्छिताः निष्कृतिः}\\
     {\large क्त्वा क्ततवतु निष्ठा सक्तुमिव तितउना}\\
     {\Large होहौऽहंसशष कार्स्न्य विद्या सङ्कर्षण}\\
     {\huge हो हौ ऽ हं स श ष कार्स्न्य विद्या सङ्कर्षण}
     }
    \end{itemize}
\end{Large}
\pagebreak

\section{Devanagari - Awesome font with Google input}
\begin{Large}
    \begin{itemize}
        \item Works with google input tool
        \item Begin writing with begin command in bracket sanskrit, hindi, etc.
        \item must use packages - polyglossia, fontawesome and set command for language used - marathi, hindi, etc. fontfamily command
        %\usepackage{polyglossia}
        %\usepackage{fontawesome}
        %\setmainlanguage{english}
        %\setotherlanguages{marathi,hindi,sanskrit}
        %\newfontfamily\devanagarifont[Script=Devanagari]{Noto Serif Devanagari}
        \item \begin{sanskrit}
\scriptsize {सर्वे मानवाः स्वतन्त्राः समुत्पन्नाः वर्तन्ते अपि च, गौरवदृशा अधिकारदृशा च समानाः एव वर्तन्ते।}\\
\small लभेत सिकतासु तैलमपि यत्नतः पीडयन्\\
\large पिबेच्च मृगतृष्णिकासु सलिलं पिपासार्दितः।\\
\Large कदाचिदपि पर्यटन् शशविषाणमासादयेन्\\
\huge न तु प्रतिनिविष्टमूर्खजनचित्तमाराधयेत्।।
\end{sanskrit}
    \end{itemize}
\end{Large}
\pagebreak

\section{Roman Transliteration - IAST}
\begin{Large}
    \begin{itemize}
    \item Needs keyboard setting changed to English (India) or can copy from Aksharmukh.\\
    Click on the "ENG" at the right corner of the taskbar. Then you will see the "Language Preference" option. Click it. Settings page opens. There click on ``add a preferred language". Select "English (India) language pack." Install it. Now, from the taskbar, click on Eng. Select Eng (India) keyboard.
    \item ā, Ā, ī, Ī, ū, Ū, ḥ, Ḥ, ṁ, Ṁ, ṭ, Ṭ, ḍ, Ḍ, ṇ, Ṇ, ś, Ś, r̥, R̥, l̥, L̥ - for these letters, press right Alt key+the respective letter.

For capital letters, press the Caps Lock key as usual.

ṣ, Ṣ - right Alt key + x, right Alt key + X 

ñ, Ñ - right Alt key + y, right Alt key + Y

ṅ, Ṅ - right Alt key + g, right Alt key + G
\item For Eg. -\\
\scriptsize ahaṁ brahmāsmi\\
\small sarvaṁ khlvidaṁ brahma\\
\large prajñānaṁ brahma\\
\Large tattvamasi\\
\huge ayamātmā brahma
\end{itemize}
\end{Large}

\pagebreak

\section{Old method}
\Large Needs pre-processing.\\
{\skt .sa;vRa;Da;ma;Ra;n,a :pa;i8a:=+tya:j1ya ma;a;mea;k\ZH{-12}{M} Za:=+NMa
v.ra:ja \ZS{12}@A \\
 A;h\ZH{-6}{M} tva;Ma .sa;vRa;pa;a;pea;Bya;ea ma;ea;[a;Y4a;ya;Sya;a;Y6a;ma
ma;a Zua;.caH\ZS{4} \ZS{12}@A\ZS{6}@A 66 \ZS{12}@A\ZS{6}@A}
\end{document}
